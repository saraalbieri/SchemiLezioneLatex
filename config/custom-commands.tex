%======================================
% Creates an underlined non-numbered section header.
\makeatletter
\makeatother

%----------------------------------------------------------------
%--> \customsection: makes unumbered small caps section heading.|
%----------------------------------------------------------------
\titleformat*{\section}{\bfseries\large} 
\newcommand{\customsection}[1]{
	\section{\textsc{#1}}\label{sec:#1} 
	% Reduce the vertical space below section headings.
	\vspace{-0.1cm} 
}
%--------------------------------------------------------
%--> \customhrule: makes a customized rule whose width  | 
%                  should be passed as parameter.       |
%--------------------------------------------------------
\newcommand{\customhrule}[1]{
	\rule[1.4pt]{\linewidth}{#1}
}
%------------------------------------------------------
%--> \doublerule: makes a double rule.                |
%------------------------------------------------------ 
\newcommand{\doublerule}[1][.4pt]{
	\noindent
	\makebox[0pt][l]{\rule[.7ex]{\linewidth}{#1}}%
	\rule[1pt]{\linewidth}{#1}\par} 
%===== Custom Ruler commands  ==================
\renewcommand{\headrulewidth}{1pt}
\renewcommand{\footrulewidth}{0.4pt}
% Disable spaces between list items in a labeled list.
\setlist{noitemsep}

%======= Numbered list: non-filled circle list ======================= Tenere
% ➀
\newlist{numberedEmptyList}{itemize}{9}
\setlist[numberedEmptyList]{topsep=4pt,partopsep=0pt,itemsep=3pt,parsep=0pt,labelindent=0.5cm,leftmargin=*}
\setlist[numberedEmptyList,9]{label=\ding{182}}

%=================================================================================================
% Command for styling tabled row header (left, center or right)
% Usage example: \thead{<Header text 1>} & \thead{<Header 2>} & \thead{<Header 3>} & \thead{<Header 4>} 
\newcommand*{\thead}[1]{\multicolumn{1}{l}{\bfseries #1}}	

%--------------------------------------------------
% ==== Doc header and footer setup.               |
%-------------------------------------------------- 
\renewcommand{\thefootnote}{\fnsymbol{footnote}}
\fancyhf{}
%- Disable the horizontal ruler in the header section.
\renewcommand{\headrulewidth}{0pt}
\chead{}
\rfoot{\fancyplain{}{pagina \thepage\ di \pageref{LastPage}}}
\cfoot{}
\lfoot{{\tiny{ \coursetitle} }}
%- TODO: move the header content here.
\fancyfoot[RO,LE] {{\tiny{pagina \thepage\ di \pageref{LastPage} }}} %RO=right odd, LE=left even 
\thispagestyle{plain}
\pagestyle{fancy}
%------------------------------------------------------------

\newcolumntype{R}[1]{>{\raggedleft\arraybackslash}p{#1}}

%-- Spacing commands ------ 
\newcommand{\vspbpara}{\vspace*{.09in}}    
\newcommand{\customvspace}{\vspace{.5cm}}    
\titlespacing{\section}{0pt}{12pt}{9pt}
%-----
\newcommand{\vtitlespacing}{\vskip 0.3cm}
%\newcommand{\paragraphentry}[1]{\noindent \textbf{\Large \underline{#1}} }


% Information boxes
\newcommand*{\info}[4][16.3]{%
  \node [ annotation, #3, scale=0.65, text width = #1em,inner sep = 2mm ] at (#2) {%
  \list{$\bullet$}{\topsep=0pt\itemsep=0pt\parsep=0pt
    \parskip=0pt\labelwidth=8pt\leftmargin=8pt
    \itemindent=0pt\labelsep=2pt}%
    #4
  \endlist
  };
}


%Refernces
\bibliographystyle{plainnat}